
\documentclass[12,a4paper]{article}

%packages
\usepackage{amsmath}

%my macros
\newcommand{\dd}[1]{\partial_{#1}}
\newcommand{\scxx}{\delta c_{xx}}
\newcommand{\scyy}{\delta c_{yy}}
\newcommand{\sczz}{\delta c_{zz}}
\newcommand{\scxy}{\delta c_{xy}}
\newcommand{\scxz}{\delta c_{xz}}
\newcommand{\scyz}{\delta c_{yz}}
\newcommand{\Wi}{\frac{1}{W_{i}}}

\begin{document}
\title{Linear stability equations}
\maketitle

\noindent For the Boundary condition on V:
\begin {align}
    V(+1) &= \dd{x} \psi(+1) = nkBTOP \cdot \widetilde{\psi_{n}} \\
	  &= 1 \\
    V(-1) &= \dd{x} \psi(-1) = nkBBOT \cdot \widetilde{\psi_{n}} \\
	  &= -1 \\
    BTOP_{m}  &= (1)^m \\
    BBOT_{m}  &= (-1)^m
\end {align}
for each Fourier mode.

For the Boundary condition on U:
\begin{align}
    U(+1) &= -\dd{y} \psi(+1) = KTOP \cdot \widetilde{\psi_{n}} \\
          &= 1\\
    U(-1) &= -\dd{y} \psi(-1) = KBOT \cdot \widetilde{\psi_{n}} \\
          &= -1 \\
    KTOP_{j}  &= BTOP_{i} \cdot MDY_{i,j} \\
    KBOT_{j}  &= BBOT_{i} \cdot MDY_{i,j}
\end{align}
for each Fourier mode, n.

To set only the imaginary part to zero to fix the phase factor:
\begin{align}
    j &= 3(2N+1)M + M-5 \\
    SPEEDCONDITION &= \delta(j) - \delta(N+j) 
\end{align}
Then I set the final row of the jacobian with this vector. I also set the final element of the residuals vector to zero. My intention was to use a Fourier component - its complex conjugate to set the imaginary part to zero. But I am not sure these two components are complex conjugates of one another?



\end{document}
